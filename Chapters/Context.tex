\section{Context and goals}
\label{section:context}
This study is set out in the context of EDF numerical tools development. EDF's R\&D National Laboratory for Hydraulics and Environment department (LNHE) uses the TELEMAC-MASCARET system to simulate complex hydro-environmental phenomenons (such as dam breaks and flooding risks) in order to anticipate the risks related to electrical production. 

TELEMAC-MASCARET results are therefore expected to produce highly reliable results. However, a great number of parameters used in these studies, specifically in morphodynamic simulations, can be set by the user and are uncertain. In order to determine the uncertain parameters impact on the system's result, an uncertainty quantification study is necessary.

In this context, an uncertainty quantification study is conducted in the morphodynamic simulation module SISYPHE. To make this study possible, an API (Application Program Interface) is implemented and coupled to TELEMAC2D, to guarantee the inter-operability with SISYPHE.
%---------------------------------------------------------------------------
