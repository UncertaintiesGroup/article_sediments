\section{Conclusion and perspectives}
In this study, an uncertainty quantification of a morphodynamic problem has been proposed.

In a sensitivity analysis step, differences were observed between a real case and an experimental case. In fact, for the real bifurcation case, the diameter was the only influencing parameter, which is not the same for the channel. 

In order to analyse the influence of sediments diameter on the model's response, an uncertainty propagation study was conducted, considering as an only uncertain parameter the sediments diameter. This study has shown that the diameter has highly propagated uncertainties where there is movement. In fact, high variances were observed in maximum erosion points for the channel, and in a deposition zone for the bifurcation. 

Finally, correlations were added and increased the variances significantly. An ANCOVA method was implemented in order to conduct a sensitivity analysis. The gap between the variables' influences decreased and variables that seemed first none-influencing (sediments diameter and transport coefficient of the Meyer-Peter and Müller formula) became of considerable influence when adding correlations. 

This study can be generalized to other applications, such as the use of different sediment transport formulas, the study of suspended sediment transport or the influence of different physical phenomenons, for example waves (TOMAWAC module in the TELEMAC-MASCARET system).
